%%
% NCHU Bachelor Thesis Template
%
% 南昌航空大学毕业设计(论文) —— 使用 XeLaTeX 编译
%
% Copyright 2020 Hurley Huang
%
% The Current Maintainer of this work is Hurley Huang.
%
% Compile with: xelatex -> biber -> xelatex -> xelatex

% 章节支持、单面打印:ctexbook
\documentclass[UTF8,AutoFakeBold,AutoFakeSlant,zihao=-4,oneside,openany]{ctexbook}
\usepackage[a4paper,left=3.5cm,right=2.5cm,top=2.5cm,bottom=2.5cm]{geometry}
% 目前 29mm 最接近 Word 排版
\usepackage{xeCJK}
\usepackage{titletoc}
\usepackage{fontspec}
\usepackage{setspace}
\usepackage{graphicx}
\usepackage{fancyhdr}
\usepackage{pdfpages}
\usepackage{setspace}
\usepackage{booktabs}
\usepackage{multirow}
\usepackage{caption}
\usepackage{tikz}
\usepackage{etoolbox}
\usepackage{hyperref}
\usepackage{xcolor}
\usepackage{caption}
\usepackage{array}
\usepackage{amsmath}
\usepackage{amssymb}
\usepackage{pdfpages}
\usepackage{float}
\usepackage[section]{placeins}

% 设置参考文献编译后端为 biber,引用格式为 GB/T7714-2015 格式
% 参考文献使用宏包见 https://github.com/hushidong/biblatex-gb7714-2015
\usepackage[
  backend=biber,
  style=gb7714-2015,
  gbalign=gb7714-2015,
  gbnamefmt=lowercase,
  gbpub=false,
  doi=false,
  url=false,
  eprint=false,
  isbn=false,
]{biblatex}

% 参考文献引用文件位于 misc/ref.bib
\addbibresource{misc/ref.bib}

% 西文字体默认为 Times New Roman
\setromanfont{Times New Roman}

% 论文中文题目
\newcommand{\thesisTitle}{毕业设计(论文)题目}
% 论文英文题目(可选)
\newcommand{\thesisTitleEN}{The Subject of Undergraduate Graduation Project (Thesis) of Nanchang Hangkong University}

% 在这里填写你的相关信息
\newcommand{\deptName}{软件学院}
\newcommand{\majorName}{软件工程}
\newcommand{\yourName}{黄XX}
\newcommand{\yourStudentID}{152012XX}
\newcommand{\mentorName}{陈XX}
% 如果你的毕设为校外毕设,请将下面这一行语句解除注释(删除第一个百分号字符)并在第二组花括号中填写你的校外毕设导师名字
% \newcommand{\externalMentorName}{左偏树}

% 主题页面格式:NCHUThesis
\fancypagestyle{NCHUThesis}{
  % 页眉高度
  \setlength{\headheight}{20pt}
  % 页码高度(不完美,比规定稍微靠下 2mm)
  \setlength{\footskip}{14pt}

  \fancyhf{}
  % 定义页眉、页码
  \fancyhead[C]{\zihao{5}\ziju{0.08}\songti{南昌航空大学学士学位论文}}
  \fancyfoot[C]{\songti\zihao{-5} \thepage}
  % 页眉分割线稍微粗一些
  \renewcommand{\headrulewidth}{0.6pt}
}

% 设置章节格式
% 一级标题:宋体,小三号,加粗;间距:段前 0.5 行,段后 1 行;
\ctexset{chapter={
    name = {第,章},
    number = {\arabic{chapter}},
    format = {\songti \bfseries \centering \zihao{-3}},
    aftername = \hspace{9bp},
    pagestyle = NCHUThesis,
    beforeskip = 8bp,
    afterskip = 32bp,
    fixskip = true,
  }
}

% 二级标题:宋体,四号,加粗;间距:段前 0.5 行,段后 0 行;
\ctexset{section={
    number = {\thechapter.\hspace{4bp}\arabic{section}},
    format = {\songti \raggedright \bfseries \zihao{4}},
    aftername = \hspace{8bp},
    beforeskip = 20bp plus 1ex minus .2ex,
    afterskip = 18bp plus .2ex,
    fixskip = true,
  }
}

% 三级标题:黑宋体、小四、加粗;间距:段前 0.5 行,段后 0 行;
\ctexset{subsection={
    number = {\thechapter.\hspace{3bp}\arabic{section}.\hspace{3bp}\arabic{subsection}},
    format = {\songti \bfseries \raggedright \zihao{-4}},
    aftername = \hspace{7bp},
    beforeskip = 17bp plus 1ex minus .2ex,
    afterskip = 14bp plus .2ex,
    fixskip = true,
  }
}

% 设置目录样式
% 添加 PDF 链接
\addtocontents{toc}{\protect\hypersetup{hidelinks}}

% 修改超链接、引用的颜色
\hypersetup{
  colorlinks=true,
  linkcolor=black,
  anchorcolor=black,
  citecolor=black
}

% 解决「目录」二字的格式问题
\renewcommand{\contentsname}{
  \fontsize{16pt}{\baselineskip}
  \normalfont\heiti{目~~~~录}
  \vspace{-8pt}
}
% 定义目录样式
\titlecontents{chapter}[0pt]{\songti \zihao{-4}}
{\thecontentslabel\hspace{\ccwd}}{}
{\hspace{.5em}\titlerule*{.}\contentspage}
\titlecontents{section}[1\ccwd]{\songti \zihao{-4}}
{\thecontentslabel\hspace{\ccwd}}{}
{\hspace{.5em}\titlerule*{.}\contentspage}
\titlecontents{subsection}[2\ccwd]{\songti \zihao{-4}}
{\thecontentslabel\hspace{\ccwd}}{}
{\hspace{.5em}\titlerule*{.}\contentspage}

% 前置页面(原创性声明、中英文摘要、目录等)
\renewcommand{\frontmatter}{
  \pagenumbering{Roman}
  \pagestyle{NCHUThesis}
}

% 正文页面
\renewcommand{\mainmatter}{
  \pagenumbering{arabic}
  \pagestyle{NCHUThesis}
}

% % 设置 caption 与 figure 之间的距离
% \setlength{\abovecaptionskip}{11pt}
% \setlength{\belowcaptionskip}{9pt}

% % 设置图片的 caption 格式
% \renewcommand{\thefigure}{\thechapter-\arabic{figure}}
% \captionsetup[figure]{font=small,labelsep=space}

% % 设置表格的 caption 格式和 caption 与 table 之间的垂直距离
% \renewcommand{\thetable}{\thechapter-\arabic{table}}
% \captionsetup[table]{font=small,labelsep=space,skip=2pt}

%%% ---- 图表标题设置 ----- %%%
\RequirePackage[labelsep=quad]{caption}     % 序号之后空一格写标题
% 设置表格标题字体为黑体, 设置图标题字体为宋体
\DeclareCaptionFont{heiti}{\heiti}
\captionsetup[table]{textfont=heiti}
\renewcommand\figurename{\songti\zihao{-4} 图}  
\renewcommand\tablename{\heiti\zihao{-4} 表} 

% 使用tabularx创建占满宽度的表格
\RequirePackage{tabularx, makecell}
\newcolumntype{L}{X}
\newcolumntype{C}{>{\centering \arraybackslash}X}
\newcolumntype{R}{>{\raggedleft \arraybackslash}X}

\RequirePackage{longtable}  % 做长表格的包
\RequirePackage{booktabs}   % 做三线表的包

% 列表样式
\RequirePackage{enumerate, enumitem}
\setlist{noitemsep}

% 调整底层 TeX 排版引擎参数以保证所有段落能够很好地以两端对齐的方式呈现
\tolerance=1
\emergencystretch=\maxdimen
\hyphenpenalty=10000
\hbadness=10000

% 设置数学公式编号格式
\renewcommand{\theequation}{\arabic{chapter}.\arabic{equation}}

\newcommand{\unnumchapter}[1]{
  \chapter*{\vskip 10bp\textmd{#1} \vskip -6bp}
  \addcontentsline{toc}{chapter}{#1}
  \stepcounter{chapter}
}

% 公式引用使用中文括号
\renewcommand{\eqref}[1]{\textup{{\normalfont(\ref{#1})\normalfont}}}

%%% ---- 引入宏包 ----- %%%
\RequirePackage{amsmath, amssymb}
\RequirePackage[amsmath,thmmarks]{ntheorem}  % 定理
\RequirePackage{graphicx, subcaption}
\RequirePackage{listings}                    % 代码段
% \RequirePackage{minted}                    % 代码高亮(需要 python 安装 pygments 库)
\RequirePackage[ruled,vlined]{algorithm2e}
\RequirePackage{algorithmic}    % 算法代码
\RequirePackage{tikz, pgfplots}              % 绘图
\RequirePackage{fontspec, color, url, array}

\RequirePackage{txfonts}                     % Times 风格(数学)字体

%%% ---- 定义字体 ----- %%%
\renewcommand{\normalsize}{\zihao{-4}}         % 正常字号
% 设置英文字体为 Times New Roman
\setmainfont[Ligatures=Rare]{Times New Roman}
\setsansfont[Ligatures=Rare]{Times New Roman}
\setmonofont[Ligatures=Rare]{Times New Roman}

% 算法两字用中文显示
\renewcommand{\algorithmcfname}{算法}

\lstdefinestyle{code}{
	backgroundcolor=\color{gray!10},   
	commentstyle=\color{green!50!black},
	keywordstyle=\color{blue},
	stringstyle=\color{magenta},
	basicstyle=\linespread{1}\footnotesize\ttfamily,
	numberstyle=\tiny,
	breakatwhitespace=false,         
	breaklines=true,                 
	captionpos=t,   
	frame=single,
	keepspaces=true,         
	language=java,        
	numbers=none,             
	numbersep=5pt,                  
	showspaces=false,                
	showstringspaces=false,
	showtabs=false,                  
	tabsize=2,
	aboveskip=1em,
	belowskip=1em,
	belowcaptionskip=12pt
}

% 修改脚注
\makeatletter%
\long\def\@makefnmark{%
\hbox {{\normalfont \textsuperscript{\circled{\@thefnmark}}}}}%
\makeatother
\makeatletter%
\long\def\@makefntext#1{%
  \parindent 1em\noindent \hb@xt@ 1.8em{\hss \circled{\@thefnmark}}#1}%
\makeatother
\skip\footins=10mm plus 1mm
\footnotesep=6pt
\renewcommand{\footnotesize}{\songti\zihao{5}}
\renewcommand\footnoterule{\vspace*{-3pt}\hrule width 0.3\columnwidth height 1pt \vspace*{2.6pt}}

\newcommand*\circled[1]{\tikz[baseline=(char.base)]{%
\node[shape=circle,draw,inner sep=0.5pt] (char) {#1};}} % 圆圈数字①

%%% ---- 数学定理样式 ----- %%%
\theoremstyle{plain}
\theoremheaderfont{\heiti}
\theorembodyfont{\songti} \theoremindent0em
\theorempreskip{0pt}
\theorempostskip{0pt}
\theoremnumbering{arabic}
%\theoremsymbol{} %定理结束时自动添加的标志
\newtheorem{theorem}{\hspace{2em}定理}[section]
\newtheorem{definition}{\hspace{2em}定义}[section]
\newtheorem{lemma}{\hspace{2em}引理}[section]
\newtheorem{corollary}{\hspace{2em}推论}[section]
\newtheorem{proposition}{\hspace{2em}性质}[section]
\newtheorem{example}{\hspace{2em}例}[section]
\newtheorem{remark}{\hspace{2em}注}[section]

\theoremstyle{nonumberplain}
\theoremheaderfont{\heiti}
\theorembodyfont{\normalfont \rm \songti}
\theoremindent0em \theoremseparator{\hspace{1em}}
\theoremsymbol{$\square$}
\newtheorem{proof}{\hspace{2em}证明}

% 文档开始
\begin{document}

% 标题页面:如无特殊需要,本部分无需改动
%%
% NCHU Bachelor Thesis Template
%
% 南昌航空大学毕业设计(论文)封面页 —— 使用 XeLaTeX 编译
%
% Copyright 2020 Hurley Huang
%
% This work may be distributed and/or modified under the
% conditions of the LaTeX Project Public License, either version 1.3
% of this license or (at your option) any later version.
% The latest version of this license is in
%   http://www.latex-project.org/lppl.txt
% and version 1.3 or later is part of all distributions of LaTeX
% version 2005/12/01 or later.
%
% This work has the LPPL maintenance status `maintained'.
%
% The Current Maintainer of this work is Spencer Woo.
%
% 封面
%
% 如无特殊需要,本页面无需更改

% Underline new command for student information
% Usage: \dunderline[<offset>]{<line_thickness>}
\newcommand\dunderline[3][-1pt]{{%
  \setbox0=\hbox{#3}
  \ooalign{\copy0\cr\rule[\dimexpr#1-#2\relax]{\wd0}{#2}}}}

% Cover Page
\begin{titlepage}
  \makeatletter
  \@ifundefined{externalMentorName}{
    % 校内毕设封面顶部间距
    \vspace*{-3mm}
  }{
    % 校外毕设封面顶部间距
    \vspace*{-3mm}
  }
  \centering

  \includegraphics[width=11.23cm]{images/header.png}

  \vspace*{26mm}

  \zihao{-0}\textbf{\ziju{0.12}\songti{毕业设计(论文)}}

  \vspace{20mm}

  \hspace{8mm}\zihao{3}\heiti\textbf{题\hspace{10mm}目:}
  \songti\zihao{-3}\selectfont{\dunderline[-10pt]{1pt}{\makebox[78mm][c]{\thesisTitle}}}

  \vspace{3mm}

  % \begin{spacing}{1.2}
  %   \zihao{3}\selectfont{\textbf{\thesisTitleEN}}
  % \end{spacing}

  \vspace{15mm}

  \flushleft

  \makeatletter
  \@ifundefined{externalMentorName}{
    % 生成校内毕设封面字段
    \makeatother
    \begin{spacing}{1.8}
      \hspace{27mm}\zihao{3}\heiti{学\hspace{11mm}院:}
      \zihao{-3}\songti\selectfont{\dunderline[-10pt]{1pt}{\makebox[78mm][c]{\deptName}}}

      \hspace{27mm}\zihao{3}\heiti{专业名称:}
      \songti\zihao{3}\selectfont{\dunderline[-10pt]{1pt}{\makebox[78mm][c]{\majorName}}}

      \hspace{27mm}\zihao{3}\heiti{班级学号:}
      \songti\zihao{-3}\selectfont{\dunderline[-10pt]{1pt}{\makebox[78mm][c]{\yourStudentID}}}

      \hspace{27mm}\zihao{3}\heiti{学生姓名:}
      \songti\zihao{-3}\selectfont{\dunderline[-10pt]{1pt}{\makebox[78mm][c]{\yourName}}}

      \hspace{27mm}\zihao{3}\heiti{指导教师:}
      \songti\zihao{-3}\selectfont{\dunderline[-10pt]{1pt}{\makebox[78mm][c]{\mentorName}}}
    \end{spacing}
  }{
    % 生成校外毕设封面字段
    \makeatother
    \begin{spacing}{1.8}
      \hspace{19.4mm}\songti\zihao{3}\selectfont{学\hspace{19.6mm}院\hspace{3mm}:\dunderline[-10pt]{1pt}{\makebox[77.4mm][c]{\deptName}}}

      \hspace{19.4mm}\songti\zihao{3}\selectfont{专\hspace{2.8mm}业\hspace{2.8mm}名\hspace{2.8mm}称\hspace{3mm}:\dunderline[-10pt]{1pt}{\makebox[77.4mm][c]{\majorName}}}

      \hspace{19.4mm}\songti\zihao{3}\selectfont{班\hspace{2.8mm}级\hspace{2.8mm}学\hspace{2.8mm}号\hspace{3mm}:\dunderline[-10pt]{1pt}{\makebox[77.4mm][c]{\yourStudentID}}}

      \hspace{19.4mm}\songti\zihao{3}\selectfont{学\hspace{2.8mm}生\hspace{2.8mm}姓\hspace{2.8mm}名\hspace{3mm}:\dunderline[-10pt]{1pt}{\makebox[77.4mm][c]{\yourName}}}

      \hspace{19.4mm}\songti\zihao{3}\selectfont{指\hspace{2.8mm}导\hspace{2.8mm}教\hspace{2.8mm}师\hspace{3mm}:\dunderline[-10pt]{1pt}{\makebox[77.4mm][c]{\mentorName}}}

      \hspace{19.4mm}\songti\zihao{3}\selectfont{校外指导教师:\dunderline[-10pt]{1pt}{\makebox[77.4mm][c]{\externalMentorName}}}
    \end{spacing}
  }

  \vspace{25mm}
  \centering
  \zihao{3}\ziju{0.5}\songti{\today}
\end{titlepage}


% 前置页面定义
\frontmatter
% 原创性声明:如无特殊需要,本部分无需改动
% 更改为 PDF 页面插入,如需要添加内容,可考虑先用 Word 制作再覆盖 misc/1_originality.pdf
\includepdf{misc/1_originality.pdf}
\newpage
% %
% NCHU Bachelor Thesis Template
%
% 南昌航空大学毕业设计(论文) —— 使用 XeLaTeX 编译
%
% Copyright 2023 Arnold Chow
%
% The Current Maintainer of this work is Arnold Chow.
%
% Compile with: xelatex -> biber -> xelatex -> xelatex
%
% 如无特殊需要,本页面无需更改

\thispagestyle{empty}

% 设置行间距
\setlength{\parskip}{0.4em}
\renewcommand{\baselinestretch}{1.41}

% 顶部空白
\vspace*{-6mm}

\begin{center}
\includegraphics[width=8cm]{images/header.png}
\end{center}
\vspace{-13mm}
% 原创性声明部分
\begin{center}
  \songti\zihao{3}\textbf{学士学位论文原创性声明}
\end{center}

% 本部分字号为小四
\zihao{-4}

本人声明,所呈交的论文是本人在导师的指导下独立完成的研究成果。
除了文中特别加以标注引用的内容外,本论文不包含法律意义上已属于他人的任何形式的研究成果,
也不包含本人已用于其他学位申请的论文或成果。
对本文的研究作出重要贡献的个人和集体,均已在文中以明确方式表明。
坚决杜绝论文买卖、代写、抄袭、剽窃等不良现象,
确保毕业设计(论文)质量。本人完全意识到本声明的法律后果由本人承担。

\vspace*{1mm}

作者签名:\hspace{40mm}日\hspace{2.5mm}期:

\vspace*{1mm}
导师签名:\hspace{40mm}日\hspace{2.5mm}期:


\vspace{17mm}

% 使用授权声明部分
\begin{center}
  \songti\zihao{3}\textbf{学位论文版权使用授权书}
\end{center}

本学位论文作者完全了解学校有关保留、使用学位论文的规定,同意学校保留并向
国家有关部门或机构送交论文的复印件和电子版,允许论文被查阅和借阅。
本人授权南昌航空大学可以将本论文的全部或部分内容编入有关数据库进行检索,
可以采用影印、缩印或扫描等复制手段保存和汇编本学位论文。
\vspace*{1mm}

作者签名:\hspace{40mm}日\hspace{2.5mm}期:

\vspace*{1mm}
导师签名:\hspace{40mm}日\hspace{2.5mm}期:

\newpage

% 摘要:在摘要相应的 TeX 文件处进行摘要部分的撰写
%
% NCHU Bachelor Thesis Template
%
% 南昌航空大学毕业设计(论文) —— 使用 XeLaTeX 编译
%
% Copyright 2023 Arnold Chow
%
% The Current Maintainer of this work is Arnold Chow.
%
% Compile with: xelatex -> biber -> xelatex -> xelatex

% 中英文摘要章节

\zihao{-4}
\vspace*{-11mm}

\begin{center}
  \heiti\zihao{3}\textbf{\thesisTitle}
\end{center}

\setcounter{page}{1}

\vspace*{-5mm}

\begin{center}
  学生姓名:\selectfont{\kaishu\makebox[28mm][l]{\yourName}}
  班级:\selectfont{\kaishu\makebox[28mm][l]{\yourClass}}  \\
  指导老师:\selectfont{\kaishu\makebox[70mm][l]{\mentorName}}
\end{center}

\vspace*{-5mm}

\setstretch{1.53}
\setlength{\parskip}{0em}

% 中文摘要正文
\noindent\zihao{4}\textbf{\heiti 摘要:}
\zihao{-4}\kaishu
论文摘要是对论文的内容不加注释和评论的简短陈述,要求扼要地说明研究工作的目的、研究方法和最终结论等,
重点是结论,是一篇具有独立性和完整性的短文。

\textcolor{blue}{摘要正文选用模板中的样式所定义的“正文”,每段落首行缩进 2 个字符;
或者手动设置成每段落首行缩进 2 个汉字,字体:楷体,字号:小四,行距:固定值 22 磅,间距:段前、段后均为 0 行。阅后删除此段。}

\textcolor{blue}{摘要是一篇具有独立性和完整性的短文,应概括而扼要地反映出本论文的主要内容。
包括研究目的、研究方法、研究结果和结论等,特别要突出研究结果和结论。
中文摘要力求语言精炼准确,本科生毕业设计(论文)摘要建议 300-500 字,一般为正文数字的5\%左右。
摘要中不可出现参考文献、图、表、化学结构式、非公知公用的符号和术语。
英文摘要与中文摘要的内容应一致。阅后删除此段。}

\vspace{4ex}

% 中文摘要关键词
\noindent\textbf{\heiti 关键词:} 南昌航空大学;本科生;毕业论文
\songti

\vspace{4ex}
\rightline{\heiti\textbf{指导老师签名:\makebox[38mm][c]{ }}}

\newpage

% 英文摘要章节
\vspace*{-2mm}

\begin{spacing}{0.95}
  \centering
  \heiti\zihao{3}\textbf{\thesisTitleEN}
\end{spacing}

\vspace*{1mm}

\begin{center}
  Student Name: \selectfont{\kaishu\makebox[38mm][l]{\yourNameEN}}
  Class: \selectfont{\kaishu\makebox[28mm][l]{\yourClass}}  \\
  Supervisor: \selectfont{\kaishu\makebox[84mm][l]{\mentorNameEN}}
\end{center}

\vspace*{-4mm}

\setstretch{1.53}
\setlength{\parskip}{0em}

% 英文摘要正文
\noindent\zihao{4}\textbf{Abstract:}
\zihao{-4} The abstract is a short statement of the content of the thesis without notes and comments, which is required to briefly explain the aim of the research work, the methodology and the final conclusions, with emphasis on the conclusions. It is a short text with independence and integrity.

\textcolor{blue}{Abstract 正文设置成每段落首行缩进 2 字符,字体:Times New Roman,字号:小四,行距:固定值 22 磅,间距:段前、段后均为 0 行。阅后删除此段。}

\vspace{4ex}
\noindent\textbf{Key Words:} NCHU; Undergraduate; Graduation Thesis

\vspace{4ex}
\rightline{\textbf{Signature of Supervisor:\makebox[38mm][c]{ }}}

\newpage

% 目录:如无特殊需要,本部分无需改动
%
% NCHU Bachelor Thesis Template
%
% 南昌航空大学毕业设计(论文) —— 使用 XeLaTeX 编译
%
% Copyright 2023 Arnold Chow
%
% The Current Maintainer of this work is Arnold Chow.
%
% Compile with: xelatex -> biber -> xelatex -> xelatex
%
% 如无特殊需要,本页面无需更改

% 目录开始
\setcounter{page}{-1}

% 调整目录行间距
\renewcommand{\baselinestretch}{1.35}
% 目录
\tableofcontents
\newpage


% 正文开始
\mainmatter
% 正文 22 磅的行距
\setlength{\parskip}{0em}
\renewcommand{\baselinestretch}{1.53}
% 修复脚注出现跨页的问题
\interfootnotelinepenalty=10000

% 第一章
%
% NCHU Bachelor Thesis Template
%
% 南昌航空大学毕业设计(论文) —— 使用 XeLaTeX 编译
%
% Copyright 2023 Arnold Chow
%
% The Current Maintainer of this work is Arnold Chow.
%
% Compile with: xelatex -> biber -> xelatex -> xelatex

% 第一章节

\chapter{一级标题}

\section{二级标题}
% 这里插入一个参考文献,仅作参考
可以通过空一行(两次回车)实现段落换行,仅仅是回车并不会产生新的段落\cite{yuFeiJiZongTiDuoXueKeSheJiYouHuaDeXianZhuangYuFaZhanFangXiang2008}。

\subsection{三级标题}

正文……\cite{simonyanVeryDeepConvolutional2015}

\section{字体字号}

{\songti \bfseries 宋体加粗} {\textbf{English}}

{\songti \itshape 宋体斜体} {\textit{English}}

{\songti \bfseries \itshape 宋体粗斜体} {\textbf{\textit{English}}}

\section{编译}

本模板必须使用 XeLaTeX + BibTeX 编译,否则会直接报错。本模板支持多个平台,结合 Sublime Text/VSCode/Overleaf 都可以使用。



% 第二章
%%
% NCHU Bachelor Thesis Template
%
% 南昌航空大学毕业设计(论文)中英文摘要 —— 使用 XeLaTeX 编译
%
% Copyright 2020 Hurley Huang
%
% The Current Maintainer of this work is Hurley Huang.

% 公式插图表格

\chapter{公式插图表格}

\section{公式的使用}

文中引用公式可以是:$a^2+b^2=c^2$。

单独一段引用公式可以是:

\begin{equation}
    \sin^2{\theta}+\cos^2{\theta}=1 \label{eq:pingfanghe}
\end{equation}

通过添加标签在正文中引用公式,如式\eqref{eq:pingfanghe}。

多行对齐的公式:

\begin{equation}
  \begin{aligned}
    f_1(x)&=(x+y)^2\\
          &=x^2+2xy+y^2
  \end{aligned}
\end{equation}

\section{插图的使用}

图示例如下:

\begin{figure}[htbp]
	\begin{center}
	    \vspace{13pt} % 调整图片与上文的垂直距离
		\includegraphics[width = 0.3\textwidth]{images/nchu_logo.png}
		\caption{图片标题} 
		\label{图片标题序号} % label 用来在文中索引
	\end{center}
\end{figure}

\section{表格的使用}

表格的输入可以考虑使用在线的工具,如~\href{https://www.tablesgenerator.com/}{Tables Generator}~能便捷地创建表格。

\subsection{普通表格}
下面是一些普通表格的示例:

\begin{table}[ht]
  \centering
  \caption{简单表格}
  \label{tab:1}
  \begin{tabular}{|l|c|r|}
    \hline
    一& 二 & 三\\
    \hline
    A& B& C\\
    \hline
  \end{tabular}
\end{table}

\begin{table}[ht]
  \centering
  \caption{简单三线表}
  \label{tab:2}
  \begin{tabular}{ccc}
    \hline
    姓名& 学号& 性别\\
    \hline
    张三& 001& 男\\
    李四& 002& 女\\
    \hline
  \end{tabular}
\end{table}

\subsection{跨页表格}

跨页表格常用于附录。

{\centering
  \begin{longtable}{ccccccccc}
  \caption{跨页表格示例} \\
  \toprule
  1     & 0 & 5  & 1  & 2  & 3  & 4  &  5 & 6 \\
  \midrule
  \endfirsthead

  \multicolumn{1}{l}{接上一页} \\
  \toprule
  1     & 0 & 5  & 1  & 2  & 3  & 4  &  5 & 6 \\
  \midrule
  \endhead

  \bottomrule
  \hline \\
  \multicolumn{9}{r}{{转下一页}} \\
  \endfoot

  \bottomrule
  \endlastfoot    

  1     & 0 & 5  & 1  & 2  & 3  & 4  &  5 & 6 \\
  1     & 0 & 5  & 1  & 2  & 3  & 4  &  5 & 6 \\
  1     & 0 & 5  & 1  & 2  & 3  & 4  &  5 & 6 \\
  1     & 0 & 5  & 1  & 2  & 3  & 4  &  5 & 6 \\
  1     & 0 & 5  & 1  & 2  & 3  & 4  &  5 & 6 \\
  1     & 0 & 5  & 1  & 2  & 3  & 4  &  5 & 6 \\
  1     & 0 & 5  & 1  & 2  & 3  & 4  &  5 & 6 \\
  1     & 0 & 5  & 1  & 2  & 3  & 4  &  5 & 6 \\
  1     & 0 & 5  & 1  & 2  & 3  & 4  &  5 & 6 \\
  1     & 0 & 5  & 1  & 2  & 3  & 4  &  5 & 6 \\
  1     & 0 & 5  & 1  & 2  & 3  & 4  &  5 & 6 \\
  1     & 0 & 5  & 1  & 2  & 3  & 4  &  5 & 6 \\
  1     & 0 & 5  & 1  & 2  & 3  & 4  &  5 & 6 \\
  1     & 0 & 5  & 1  & 2  & 3  & 4  &  5 & 6 \\
  1     & 0 & 5  & 1  & 2  & 3  & 4  &  5 & 6 \\
  1     & 0 & 5  & 1  & 2  & 3  & 4  &  5 & 6 \\
  1     & 0 & 5  & 1  & 2  & 3  & 4  &  5 & 6 \\
  1     & 0 & 5  & 1  & 2  & 3  & 4  &  5 & 6 \\
  1     & 0 & 5  & 1  & 2  & 3  & 4  &  5 & 6 \\
  
  \end{longtable}
}

\subsection{统计表格}

统计表格一般是以三线表的形式。

\begin{table}[ht]
  \centering
  \caption{统计数据表格}
  \label{tab:3}
  \begin{tabularx}{\textwidth}{CCCC}
    \toprule
    序号&年龄&身高&体重\\
    \midrule
    1&14&156&42\\
    2&16&158&45\\
    3&14&162&48\\
    4&15&163&50\\
    \cmidrule{2-4} %添加2-4列的中线
    平均&15&159.75&46.25\\
    \bottomrule
  \end{tabularx}
\end{table}

\subsection{复杂表格}

表\ref*{tab:PPer}是一个复杂三线表的样式。

\begin{table}[ht]
  \centering
  \caption{复杂三线表}
  \begin{tabular}{l c c rrrrrrr}
    \hline\hline
    Audio & Audibility & Decision & \multicolumn{7}{c}{Sum of Extracted Bits}
    \\ [0.5ex]
    \hline
    \multirow{3}{*}{Police} & \multirow{3}{*}{5} & soft & 1 & $-1$ & 1 & 1 & $-1$ & $-1$ & 1 \\
    & & mid & 2 & $-4$ & 4 & 4 & $-2$ & $-4$ & 4 \\
    & & hard & 1 & $-1$ & 1 & 1 & $-1$ & $-1$ & 1 \\
    \multirow{3}{*}{Beethoven} & \multirow{3}{*}{1} & soft & 1 & $-1$ & 1 & 1 & $-1$ & $-1$ & 1 \\
    & & mid & 1 & $-1$ & 1 & 1 & $-1$ & $-1$ & 1 \\
    & & hard & 8 & $-8$ & 2 & 8 & $-8$ & $-8$ & 6 \\
    \multirow{3}{*}{Metallica} & \multirow{3}{*}{1} & soft & 1 & $-1$ & 1 & 1 & $-1$ & $-1$ & 1 \\
    & & mid & 1 & $-1$ & 1 & 1 & $-1$ & $-1$ & 1 \\
    & & hard & 4 & $-8$ & 8 & 4 & $-8$ & $-8$ & 8 \\

    % Entering 1st row
    % & & soft & 1 & $-1$ & 1 & 1 & $-1$ & $-1$ & 1 \\[-1ex]
    % \raisebox{1.5ex}{Police} & \raisebox{1.5ex}{5} & hard
    % & 2 & $-4$ & 4 & 4 & $-2$ & $-4$ & 4 \\[1ex]
    % % Entering 2nd row
    % & & soft & 1 & $-1$ & 1 & 1 & $-1$ & $-1$ & 1 \\[-1ex]
    % \raisebox{1.5ex}{Beethoven} & \raisebox{1.5ex}{5} & hard
    % & 8 & $-8$ & 2 & 8 & $-8$ & $-8$ & 6 \\[1ex]
    % % Entering 3rd row
    % & & soft & 1 & $-1$ & 1 & 1 & $-1$ & $-1$ & 1 \\[-1ex]
    % \raisebox{1.5ex}{Metallica} & \raisebox{1.5ex}{5} & hard
    % & 4 & $-8$ & 8 & 4 & $-8$ & $-8$ & 8 \\[1ex]

    % Entering 4rd row
    % & & soft & 1 & $-1$ & 1 & 1 & $-1$ & $-1$ & 1 \\[-1ex]
    % \raisebox{1.5ex}{Test} & \raisebox{1.5ex}{1} & mid
    % & 4 & $-8$ & 8 & 4 & $-8$ & $-8$ & 8 \\[0ex] & hard
    % & 4 & $-8$ & 8 & 4 & $-8$ & $-8$ & 8 \\[1ex]


    % [1ex] adds vertical space
    \hline % inserts single-line
  \end{tabular}
  \label{tab:PPer}
\end{table}

\section{列表的使用}

\subsection{有序列表}

即有序列表。

\begin{enumerate}
    \item 第一项
        \begin{enumerate}
            \item 第一项中的第一项
            \item 第一项中的第二项
        \end{enumerate}
    \item 第二项
        \begin{enumerate}[label=(\roman*)]
            \item 第一项中的第一项
            \item 第一项中的第二项
        \end{enumerate}
    \item 第三项
\end{enumerate}

\subsection{无序列表}

即不计数列表。

\begin{itemize}
    \item 第一项
        \begin{itemize}
            \item 第一项中的第一项
            \item 第一项中的第二项
        \end{itemize}
    \item 第二项
    \item 第三项
\end{itemize}

\section{定理的使用}

\begin{theorem}
  设向量$\boldsymbol a\neq\boldsymbol 0$,那么向量$\boldsymbol b//\boldsymbol a$的充分必要条件是:存在唯一的实数$\lambda$,使$\boldsymbol b=\lambda \boldsymbol a$。
\end{theorem}
\begin{definition}
  这是一条定义。
\end{definition}
\begin{lemma}
  这是一条引理。
\end{lemma}
\begin{corollary}
  对数轴上任意一点$P$,轴上有向线段$\vec {OP}$都可唯一地表示为点$P$的坐标与轴上单位向量$\boldsymbol e_u$的乘积:$\vec {OP}=u \boldsymbol e_u$。
\end{corollary}
\begin{proposition}
  这是一条性质。
\end{proposition}
\begin{example}
  这是一条例。
\end{example}
\begin{remark}
  这是一条注。
\end{remark}
% 第三章
%
% NCHU Bachelor Thesis Template
%
% 南昌航空大学毕业设计(论文) —— 使用 XeLaTeX 编译
%
% Copyright 2023 Arnold Chow
%
% The Current Maintainer of this work is Arnold Chow.
%
% Compile with: xelatex -> biber -> xelatex -> xelatex

% 引用

\chapter{引用}

\section{脚注}

示例:

注释\footnote{解释注释}。

\section{引用参考文献}

正文……\cite{yuFeiJiZongTiDuoXueKeSheJiYouHuaDeXianZhuangYuFaZhanFangXiang2008}


% 第四章
%
% NCHU Bachelor Thesis Template
%
% 南昌航空大学毕业设计(论文) —— 使用 XeLaTeX 编译
%
% Copyright 2023 Arnold Chow
%
% The Current Maintainer of this work is Arnold Chow.
%
% Compile with: xelatex -> biber -> xelatex -> xelatex

% 代码

\chapter{代码}

\section{代码块高亮}

\begin{lstlisting}[style=code]
YourSmartContract contract = YourSmartContract.load("0x<address>|<ensName>", <web3j>, <credentials>, GAS_PRICE, GAS_LIMIT);
\end{lstlisting}

\section{算法描述/伪代码}

\begin{algorithm}[H]
  \setstretch{1.5} % 代码间行距设定
  \SetAlgoLined
  \KwResult{Write here the result }
   initialization\;
   \While{While condition}{
    instructions\;
    \eIf{condition}{
     instructions1\;
     }{
     instructions3\;
    }
   }
  \caption{How to write algorithms}
\end{algorithm}



% 结论:在结论相应的 TeX 文件处进行结论部分的撰写
%
% NCHU Bachelor Thesis Template
%
% 南昌航空大学毕业设计(论文) —— 使用 XeLaTeX 编译
%
% Copyright 2023 Arnold Chow
%
% The Current Maintainer of this work is Arnold Chow.
%
% Compile with: xelatex -> biber -> xelatex -> xelatex

% 结论

\chapter{结论}

% 结论部分尽量不使用 \subsection 二级标题,只使用 \section 一级标题

% 这里插入一个参考文献,仅作参考
本文结论……。\cite{dengImageNetLargescaleHierarchical2010}

\textcolor{blue}{结论作为毕业设计(论文)正文的最后部分单独排写,加章号。结论是对整个论文主要结果的总结。在结论中应明确指出本研究的创新点,对其应用前景和社会、经济价值等加以预测和评价,并指出今后进一步在本研究方向进行研究工作的展望与设想。结论部分的撰写应简明扼要,突出创新性。阅后删除此段。}

\textcolor{blue}{结论正文样式与文章正文相同:宋体、小四;行距:22 磅;间距段前段后均为 0 行。阅后删除此段。}

% 参考文献:如无特殊需要,参考文献相应的 TeX 文件无需改动,添加参考文献请使用 BibTeX 的格式
%   添加至 misc/ref.bib 中,并在正文的相应位置使用 \cite{xxx} 的格式引用参考文献
%
% NCHU Bachelor Thesis Template
%
% 南昌航空大学毕业设计(论文) —— 使用 XeLaTeX 编译
%
% Copyright 2023 Arnold Chow
%
% The Current Maintainer of this work is Arnold Chow.
%
% Compile with: xelatex -> biber -> xelatex -> xelatex
%
% 如无特殊需要,本页面无需更改

% 参考文献开始
\unnumchapter{参考文献}
\renewcommand{\thechapter}{参考文献}

% 设置参考文献字号为 5 号
\renewcommand*{\bibfont}{\zihao{5}}
% 设置参考文献各个项目之间的垂直距离为 0
\setlength{\bibitemsep}{0ex}
\setlength{\bibnamesep}{0ex}
\setlength{\bibinitsep}{0ex}
% 设置单倍行距
\renewcommand{\baselinestretch}{1.2}
% 设置参考文献顺序标签 `[1]` 与文献内容 `作者. 文献标题...` 的间距
\setlength{\biblabelsep}{0.5mm}
% 设置参考文献后文缩进为 0(与 Word 模板保持一致)
\renewcommand{\itemcmd}{
  \addvspace{\bibitemsep} % 恢复 \bibitemsep 的作用
  \mkgbnumlabel{\printfield{labelnumber}}
  \hspace{\biblabelsep}}

% 删除默认的「参考文献 / Reference」标题,使用上面定义的 section 标题
\printbibliography[heading=none]

% 附录:在附录相应的 TeX 文件处进行附录部分的撰写
\input{misc/5_appendix.tex}
% 致谢:在致谢相应的 TeX 文件处进行致谢部分的撰写
\input{misc/6_acknowledgements.tex}

\end{document}
