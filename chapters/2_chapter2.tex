%%
% NCHU Bachelor Thesis Template
%
% 南昌航空大学毕业设计(论文)中英文摘要 —— 使用 XeLaTeX 编译
%
% Copyright 2020 Hurley Huang
%
% The Current Maintainer of this work is Hurley Huang.

% 公式插图表格

\chapter{公式插图表格}

\section{公式的使用}

文中引用公式可以是:$a^2+b^2=c^2$。

单独一段引用公式可以是:

\begin{equation}
    \sin^2{\theta}+\cos^2{\theta}=1 \label{eq:pingfanghe}
\end{equation}

通过添加标签在正文中引用公式,如式\eqref{eq:pingfanghe}。

多行对齐的公式:

\begin{equation}
  \begin{aligned}
    f_1(x)&=(x+y)^2\\
          &=x^2+2xy+y^2
  \end{aligned}
\end{equation}

\section{插图的使用}

图示例如下:

\begin{figure}[htbp]
	\begin{center}
	    \vspace{13pt} % 调整图片与上文的垂直距离
		\includegraphics[width = 0.3\textwidth]{images/nchu_logo.png}
		\caption{图片标题} 
		\label{图片标题序号} % label 用来在文中索引
	\end{center}
\end{figure}

\section{表格的使用}

表格的输入可以考虑使用在线的工具,如~\href{https://www.tablesgenerator.com/}{Tables Generator}~能便捷地创建表格。

\subsection{普通表格}
下面是一些普通表格的示例:

\begin{table}[ht]
  \centering
  \caption{简单表格}
  \label{tab:1}
  \begin{tabular}{|l|c|r|}
    \hline
    一& 二 & 三\\
    \hline
    A& B& C\\
    \hline
  \end{tabular}
\end{table}

\begin{table}[ht]
  \centering
  \caption{简单三线表}
  \label{tab:2}
  \begin{tabular}{ccc}
    \hline
    姓名& 学号& 性别\\
    \hline
    张三& 001& 男\\
    李四& 002& 女\\
    \hline
  \end{tabular}
\end{table}

\subsection{跨页表格}

跨页表格常用于附录。

{\centering
  \begin{longtable}{ccccccccc}
  \caption{跨页表格示例} \\
  \toprule
  1     & 0 & 5  & 1  & 2  & 3  & 4  &  5 & 6 \\
  \midrule
  \endfirsthead

  \multicolumn{1}{l}{接上一页} \\
  \toprule
  1     & 0 & 5  & 1  & 2  & 3  & 4  &  5 & 6 \\
  \midrule
  \endhead

  \bottomrule
  \hline \\
  \multicolumn{9}{r}{{转下一页}} \\
  \endfoot

  \bottomrule
  \endlastfoot    

  1     & 0 & 5  & 1  & 2  & 3  & 4  &  5 & 6 \\
  1     & 0 & 5  & 1  & 2  & 3  & 4  &  5 & 6 \\
  1     & 0 & 5  & 1  & 2  & 3  & 4  &  5 & 6 \\
  1     & 0 & 5  & 1  & 2  & 3  & 4  &  5 & 6 \\
  1     & 0 & 5  & 1  & 2  & 3  & 4  &  5 & 6 \\
  1     & 0 & 5  & 1  & 2  & 3  & 4  &  5 & 6 \\
  1     & 0 & 5  & 1  & 2  & 3  & 4  &  5 & 6 \\
  1     & 0 & 5  & 1  & 2  & 3  & 4  &  5 & 6 \\
  1     & 0 & 5  & 1  & 2  & 3  & 4  &  5 & 6 \\
  1     & 0 & 5  & 1  & 2  & 3  & 4  &  5 & 6 \\
  1     & 0 & 5  & 1  & 2  & 3  & 4  &  5 & 6 \\
  1     & 0 & 5  & 1  & 2  & 3  & 4  &  5 & 6 \\
  1     & 0 & 5  & 1  & 2  & 3  & 4  &  5 & 6 \\
  1     & 0 & 5  & 1  & 2  & 3  & 4  &  5 & 6 \\
  1     & 0 & 5  & 1  & 2  & 3  & 4  &  5 & 6 \\
  
  \end{longtable}
}

\subsection{统计表格}

统计表格一般是以三线表的形式。

\begin{table}[ht]
  \centering
  \caption{统计数据表格}
  \label{tab:3}
  \begin{tabularx}{\textwidth}{CCCC}
    \toprule
    序号&年龄&身高&体重\\
    \midrule
    1&14&156&42\\
    2&16&158&45\\
    3&14&162&48\\
    4&15&163&50\\
    \cmidrule{2-4} %添加2-4列的中线
    平均&15&159.75&46.25\\
    \bottomrule
  \end{tabularx}
\end{table}

\subsection{复杂表格}

下面是一个复杂三线表的样式。

\begin{table}[H]
  \centering
  \caption{复杂三线表}
  \begin{tabular}{l c c rrrrrrr}
    \hline\hline
    Audio & Audibility & Decision & \multicolumn{7}{c}{Sum of Extracted Bits}
    \\ [0.5ex]
    \hline
    \multirow{3}{*}{Police} & \multirow{3}{*}{5} & soft & 1 & $-1$ & 1 & 1 & $-1$ & $-1$ & 1 \\
    & & mid & 2 & $-4$ & 4 & 4 & $-2$ & $-4$ & 4 \\
    & & hard & 1 & $-1$ & 1 & 1 & $-1$ & $-1$ & 1 \\
    \multirow{3}{*}{Beethoven} & \multirow{3}{*}{1} & soft & 1 & $-1$ & 1 & 1 & $-1$ & $-1$ & 1 \\
    & & mid & 1 & $-1$ & 1 & 1 & $-1$ & $-1$ & 1 \\
    & & hard & 8 & $-8$ & 2 & 8 & $-8$ & $-8$ & 6 \\
    \multirow{3}{*}{Metallica} & \multirow{3}{*}{1} & soft & 1 & $-1$ & 1 & 1 & $-1$ & $-1$ & 1 \\
    & & mid & 1 & $-1$ & 1 & 1 & $-1$ & $-1$ & 1 \\
    & & hard & 4 & $-8$ & 8 & 4 & $-8$ & $-8$ & 8 \\

    % Entering 1st row
    % & & soft & 1 & $-1$ & 1 & 1 & $-1$ & $-1$ & 1 \\[-1ex]
    % \raisebox{1.5ex}{Police} & \raisebox{1.5ex}{5} & hard
    % & 2 & $-4$ & 4 & 4 & $-2$ & $-4$ & 4 \\[1ex]
    % % Entering 2nd row
    % & & soft & 1 & $-1$ & 1 & 1 & $-1$ & $-1$ & 1 \\[-1ex]
    % \raisebox{1.5ex}{Beethoven} & \raisebox{1.5ex}{5} & hard
    % & 8 & $-8$ & 2 & 8 & $-8$ & $-8$ & 6 \\[1ex]
    % % Entering 3rd row
    % & & soft & 1 & $-1$ & 1 & 1 & $-1$ & $-1$ & 1 \\[-1ex]
    % \raisebox{1.5ex}{Metallica} & \raisebox{1.5ex}{5} & hard
    % & 4 & $-8$ & 8 & 4 & $-8$ & $-8$ & 8 \\[1ex]

    % Entering 4rd row
    % & & soft & 1 & $-1$ & 1 & 1 & $-1$ & $-1$ & 1 \\[-1ex]
    % \raisebox{1.5ex}{Test} & \raisebox{1.5ex}{1} & mid
    % & 4 & $-8$ & 8 & 4 & $-8$ & $-8$ & 8 \\[0ex] & hard
    % & 4 & $-8$ & 8 & 4 & $-8$ & $-8$ & 8 \\[1ex]


    % [1ex] adds vertical space
    \hline % inserts single-line
  \end{tabular}
  \label{tab:PPer}
\end{table}

\section{列表的使用}

\subsection{有序列表}

即有序列表。

\begin{enumerate}
    \item 第一项
        \begin{enumerate}
            \item 第一项中的第一项
            \item 第一项中的第二项
        \end{enumerate}
    \item 第二项
        \begin{enumerate}[label=(\roman*)]
            \item 第一项中的第一项
            \item 第一项中的第二项
        \end{enumerate}
    \item 第三项
\end{enumerate}

\subsection{无序列表}

即不计数列表。

\begin{itemize}
    \item 第一项
        \begin{itemize}
            \item 第一项中的第一项
            \item 第一项中的第二项
        \end{itemize}
    \item 第二项
    \item 第三项
\end{itemize}

\section{定理的使用}

\begin{theorem}
  设向量$\boldsymbol a\neq\boldsymbol 0$,那么向量$\boldsymbol b//\boldsymbol a$的充分必要条件是:存在唯一的实数$\lambda$,使$\boldsymbol b=\lambda \boldsymbol a$。
\end{theorem}
\begin{definition}
  这是一条定义。
\end{definition}
\begin{lemma}
  这是一条引理。
\end{lemma}
\begin{corollary}
  对数轴上任意一点$P$,轴上有向线段$\vec {OP}$都可唯一地表示为点$P$的坐标与轴上单位向量$\boldsymbol e_u$的乘积:$\vec {OP}=u \boldsymbol e_u$。
\end{corollary}
\begin{proposition}
  这是一条性质。
\end{proposition}
\begin{example}
  这是一条例。
\end{example}
\begin{remark}
  这是一条注。
\end{remark}